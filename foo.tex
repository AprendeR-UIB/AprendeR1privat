
# Logística de `R` {# sec:paquetes}


 Muchas funciones y tablas de datos útiles no vienen con la instalación básica de `R`, sino que forman parte de `paquetes` \index{paquete} (*packages*), que se tienen que instalar y cargar para poderlos usar. 
Por citar un par de ejemplos, el paquete `magic`\index{paquete!{\tt magic}} lleva una función `magic`\indR{magic} que crea {cuadrados mágicos};^[ Un `cuadrado mágico`  es una tabla cuadrada de números naturales, diferentes dos a dos, tal que las sumas de todas sus columnas,  de todas sus filas y  de sus dos diagonales principales valen todas lo mismo.]  para usarla, tenemos que instalar y cargar este paquete. De manera similar,  el paquete `ggplot2` incorpora una serie  de funciones para dibujar gráficos avanzados que no podemos usar si primero no instalamos y cargamos este paquete.

Podemos consultar en la pestaña  **Packages**  la lista de paquetes que tenemos instalados. Los paquetes que aparecen marcados en esta lista son los que  tenemos  cargados en la sesión actual. Si queremos cargar un paquete ya instalado, basta marcarlo en esta lista; podemos hacerlo también desde la consola, con la instrucción

`library(``paquete` `)`.


En caso de necesitar un paquete que no tengamos instalado, hay que instalarlo antes de poderlo cargar.
La mayoría de los paquetes se pueden instalar desde el repositorio del CRAN;
esto se puede hacer de dos maneras:

* Desde la consola, entrando la instrucción

`install.packages("``paquete` `", dep=TRUE)`

(las comillas son obligatorias). El parámetro `dep=TRUE` obliga a `R` a instalar no sólo el paquete requerido, sino todos aquellos de los que dependa para funcionar correctamente. 

* Pulsando el botón <<*Install*>> de la barra superior de la pestaña de paquetes; al hacerlo, `Rstudio` abre una ventana dónde se nos pide el nombre del paquete a instalar. Conviene dejar marcada la opción <<*Install dependencies*>>, para que se instalen también los paquetes necesarios para su funcionamiento.

Así, supongamos que queremos construir cuadrados mágicos, pero aún no hemos cargado el paquete `magic`.
```
> magic(10)
Error: could not find function "magic"
> install.packages("magic", dep=TRUE) #Instalamos el paquete magic; también lo podéis hacer a través de la ventana de paquetes
...
> library(magic)  #Cargamos el paquete; también lo podéis hacer a través de la ventana de paquetes
...
> magic(10)
      [,1] [,2] [,3] [,4] [,5] [,6] [,7] [,8] [,9] [,10]
 [1,]   34   35    6    7   98   99   70   71   42    43
 [2,]   36   33    8    5  100   97   72   69   44    41
 [3,]   11   10   83   82   75   74   47   46   39    38
 [4,]   12    9   84   81   73   76   48   45   40    37
 [5,]   87   86   79   78   51   50   23   22   15    14
 [6,]   85   88   77   80   52   49   21   24   13    16
 [7,]   63   62   55   54   27   26   19   18   91    90
 [8,]   61   64   53   56   25   28   17   20   89    92
 [9,]   59   58   31   30    3    2   95   94   67    66
[10,]   57   60   29   32    1    4   93   96   65    68
```

Cuando cerramos `Rstudio`, los paquetes instalados en la sesión siguen instalados, pero cargados se pierden; por lo tanto, si queremos volver a usarlos en otra sesión, tendremos que volver a cargarlos.

Hay paquetes que no se encuentran en el CRAN y que, por lo tanto, no se pueden instalar de la forma que hemos visto. Cuando sea necesario, ya explicaremos la manera de instalarlos y cargarlos en cada caso. 





